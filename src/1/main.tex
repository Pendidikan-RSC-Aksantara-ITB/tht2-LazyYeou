\documentclass{article}
\usepackage[indonesian]{babel}

\usepackage[letterpaper,top=2cm,bottom=2cm,left=3cm,right=3cm,marginparwidth=1.75cm]{geometry}
\usepackage[colorlinks=true, allcolors=blue]{hyperref}

\title{Essai}
\author{Muhammad Ridwan Nasir Firdaus}

\begin{document}
\maketitle


Ketertarikan saya dengan dunia teknologi sudah bermula sejak saya masih duduk di tingkat sekolah dasar lebih tepatnya saat kelas tiga sekolah dasar. Karena saya sangat menggemari \textit{Video Game}, terbesit dipikiran saya tentang "Gimana sih cara ngebuat \textit{game}?". Dari sana saya mulai untuk mencari-cari \textit{tutorial} untuk \textit{develop video game}, saya memulai dari menggunakan \textit{software} Microsoft Power Point (serius ini bisa), Construct, dan RPG Maker. Sampai akhirnya saya mulai bosan dengan dunia \textit{game development}, akhirnya saya memutuskan untuk tidak melanjutkannya dan beralih ke bidang animasi dan 3D \textit{design} . Minat dalam dunia \textit{programming} itu hilang sampai saya berada di jenjang SMA, dimana saat itu saya ditunjuk sebagai leader untuk men-\textit{develop} sebuah aplikasi dan sejak saat itu minat saya di dunia teknologi khususnya untuk mengembangkan suatu \textit{product} berbasis teknologi tumbuh kembali.

Aksantara sendiri merupakan salah satu UKM yang ingin saya ikuti sedari awal saya masuk ke ITB hal itu dikarenakan dulu saya pernah memiliki ketertarikan dengan mobil RC dan drone ditambah dengan \textit{demand} industri otomasi saat ini yang semakin tinggi sehingga penggunaan UAV akan sangat diperlukan diberbagai bidang di masa depan. Oleh karena itu saya mendaftar dalam Aksantara, di samping saya ingin mengembangkan skill saya disini. Saya memilih departemen \textit{Robotics Software Control} (RSC) karena RSC merupakan departemen yang mengaplikasikan ilmu pemrograman yang paling banyak dibanding departemen-departemen lainnya dan hal itu juga selaras dengan jurusan saya saat ini (Teknik Informatika). Proses pendidikan selama ini sangat membantu dalam meningkatkan pengetahuan saya, melalui \textit{Take Home Test}(THT) yang diberikandengan \textit{deadline} yang cukup singkat memaksa saya untuk belajar serta adaptasi dengan cepat, hal itu juga melatih saya untuk bekerja dibawah tekanan waktu.

Jika lolos dalam seleksi RSC Aksantara ITB 2026 saya akan berdedikasi dalam tim yang saya masuki nantinya (semoga VTOL). Memenangkan Kontes Robot Terbang Indonesia (KRTI) bersama tim merupakan target terbesar saya selama di Aksantara, namun target saya tidak hanya dalam prestasi. Saya juga ingin menambah kenalan dengan teman seangkatan juga dengan kakak-kakak angkatan atas. Serta ingin terus belajar hal-hal baru, baik itu teknikal ataupun non-teknikal. Harapan saya jika lolos dalam seleksi Aksantara yaitu ingin menjadikan Aksantara sebagai tempat di mana saya bisa terus berkembang, berkontribusi secara nyata, dan mengaplikasikan ilmu khususnya pemrograman demi kemajuan teknologi di Indonesia

\end{document}
