\documentclass[indonesian]{article}

\usepackage[provide=*]{babel}
\usepackage[letterpaper,top=2cm,bottom=2cm,left=3cm,right=3cm,marginparwidth=1.75cm]{geometry}
\usepackage{biblatex}

\usepackage{amsmath}
\usepackage{graphicx}
\usepackage[colorlinks=true, allcolors=blue]{hyperref}

\title{Ground Control Station}
\author{Muhammad Ridwan Nasir Firdaus}

\begin{document}
\maketitle

\section{Single Page Application}

\textit{Single Page Application}(SPA) merupakan konsep aplikasi web yang hanya memiliki satu halaman. Web dengan konsep SPA tidak memerlukan reload halaman web setelah melakukan \textit{request} kepada \textit{server} seperti web biasa (\textit{Multi Page Application}), melainkan Javascript API hanya perlu melakukan fetch ketika konten ingin ditampilkan. Kelebihan dari SPA, yaitu waktu \textit{load} web menjadi lebih cepat, tidak perlu \textit{query} tambahan ke \textit{server}, halaman web responsif, dan meningkatkan \textit{user-experience}. Namun, konsep ini masih memiliki beberapa kekurangan, seperti SEO pada web akan lebih sulit untuk dioptimasi dan user memerlukan Javascript engine untuk menjalankan web SPA.

\section{Service Dalam Backend}
Dalam \textit{Backend} service adalah program/\textit{software application} yang berjalan dalam server dan bertangguna jawab untuk nge-\textit{handle} \textit{business logic}, autentikasi, interaksi \textit{database}, dan fungsi inti lainnya. Beberapa kegunaan \textit{service} dalam membantu modularitas seperti dalam pemisahan tanggung jawab antara \textit{router} dengan {service}, jadi \textit{routes} hanya menangani HTTP dan \textit{routes} menangani \textit{bussiness logic}-nya, jadi ketika \textit{logic} diubah maka \textit{endpoint} tidak perlu disentuh dan meningkatkan \textit{readability} dari program.  


\end{document}
